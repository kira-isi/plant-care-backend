%%%%%%%%%%%%%%%%%%%%%%%%%%%%%%%%%%%%%%%%%%%%%%%%%%%%%%%%%%%%%%%%%%%%%%%%%%%%%%%

%%%%%%%%%%%%%%%%%%%%%%%%%%%%%%%%%%%%%%%%%%%%%%%%%%%%%%%%%%%%%%%%%%%%%%%%%%%%%%%

\documentclass[
%   ngerman          % neue deutsche Rechtschreibung
  ,a4paper          % Papiergrösse
% ,twoside          % Zweiseitiger Druck (rechts/links)
% ,10pt             % Schriftgrösse
% ,11pt
  ,12pt
  ,pdftex
%  ,disable         % Todo-Markierungen auschalten
]{report}

% Bitte die Codierung Ihrer Dateien auswählen:
% \usepackage[latin1]{inputenc}    % Für UNIX mit ISO-LATIN-codierten Dateien
% \usepackage[applemac]{inputenc}  % Für Apple Mac
% \usepackage[ansinew]{inputenc}   % Für Microsoft Windows
\usepackage[utf8]{inputenc}        % UTF-8 codierte Dateien
                                   % Dieses Dokument ist unter Unix erstellt, daher
                                   % wird diese Input-Codierung benutzt.


\usepackage{bericht}
\usepackage{acronym}
\usepackage{listings}
\usepackage[sfdefault]{arimo}
\usepackage[ngerman]{babel}
\usepackage[T1]{fontenc}
\usepackage{inconsolata}
\usepackage[onehalfspacing]{setspace}
\usepackage{microtype}				% verbesserter Randausgleich
\usepackage[square, numbers]{natbib}        % natbib als Bibliothek
\bibliographystyle{unsrtnat}                % Nummerische Stil Zitate werden nur mit einer Nummer referenziert
%       unsrtnat, abbrvnat                  % Numerische Stile sortiert und unsortiert
% mehr Infos zu natbib:  https://www.overleaf.com/learn/latex/bibliography_management_with_natbib
\usepackage{nonfloat}               % sollte keine Gleitumgebung geben

%%%%%%%%%%%%%%%%%%%%%%%%%%%%%%%%%%%%%%%%%%%%%%%%%%%%%%%%%%%%%%%%%%%%%%%%%%%%%%%
%% Angaben zur Arbeit
%%%%%%%%%%%%%%%%%%%%%%%%%%%%%%%%%%%%%%%%%%%%%%%%%%%%%%%%%%%%%%%%%%%%%%%%%%%%%%%

\newcommand{\Autoren}{Kira Isenberg, Fabian Hagenmeier}
\newcommand{\MatrikelNummer}{3427985, 2207087}
\newcommand{\Kursbezeichnung}{TINF22B1}

%%%%%%%%%%%%%%%%%%%%%%%%%%%%%%%%%%%%%%%%%%%%%%%%%%%%%%%%%%%%%%%%%%%%%%%%%%%%%%%%%%%%%

\newcommand{\Was}{Backend für eine Pflanzenpflege-Management-Plattform}
% Wird auf dem Deckblatt in der Erklärung benutzt

%%%%%%%%%%%%%%%%%%%%%%%%%%%%%%%%%%%%%%%%%%%%%%%%%%%%%%%%%%%%%%%%%%%%%%%%%%%%%%%%%%%%%

\newcommand{\Titel}{Programmentwurf - ASWE}
\newcommand{\AbgabeDatum}{31.05.2025}

\newcommand{\Studiengang}{Informatik}

\newcommand{\Studienrichtung}{Medizinische Informatik}
% \newcommand{\Studiengang}{Angewandte Informatik}

%hyperhyper
%howmuchisthefish
\hypersetup{%%
	pdfauthor={\Autoren},
	pdftitle={\Titel},
	pdfsubject={\Was}
}

%%%%%%%%%%%%%%%%%%%%%%%%%%%%%%%%%%%%%%%%%%%%%%%%%%%%%%%%%%%%%%%%%%%%%%%%%%%%%%%
%%%%%%%%%%%%%%%%%%%%%%%%%%%%%%%%%%%%%%%%%%%%%%%%%%%%%%%%%%%%%%%%%%%%%%%%%%%%%%%
\begin{document}
\pagenumbering{roman}
%%%%%%%%%%%%%%%%%%%%%%%%%%%%%%%%%%%%%%%%%%%%%%%%%%%%%%%%%%%%%%%%%%%%%%%%%%%%%%%
% Titelseite, Hier muss man nichts ändern

\begin{titlepage}
\begin{center}
	\vspace*{-2cm}
	\hfill\includegraphics[width=4cm]{img/dhbw-logo}\\[2cm]
	{\Huge \Titel}\\[1.5cm]
	{\Huge\scshape \Was}\\[1.5cm]
	{\large Studiengang \Studiengang}\\[0.5cm]
	{\large an der}\\[0.5cm]
	{\large Dualen Hochschule Baden-Württemberg Karlsruhe}\\[0.5cm]

	\vfill
\end{center}
\begin{tabular}{l@{\hspace{2cm}}l}
Studenten	                    & \Autoren		\\
Matrikelnummern	                & \MatrikelNummer		\\
Kurs			         		& \Kursbezeichnung		\\
Abgabedatum               		& \AbgabeDatum		    \\
Dozent			         		& Daniel Lindner		\\

\end{tabular}
\end{titlepage}
\onehalfspacing

%%%%%%%%%%%%%%%%%%%%%%%%%%%%%%%%%%%%%%%%%%%%%%%%%%%%%%%%%%%%%%%%%%%%%%%%%%%%%%%

% Nur für Bachelorarbeiten einfügen:
% hab ich immer drin :)
%%%%%%%%%%%%%%%%%%%%%%%%%%%%%%%%%%%%%%%%%%%%%%%%%%%%%%%%%%%%%%%%%%%%%%%%%%%%%%%
%% Descr:       Vorlage für Berichte der DHBW-Karlsruhe, Erklärung
%% Author:      Prof. Dr. Jürgen Vollmer, vollmer@dhbw-karlsruhe.de
%% $Id: erklaerung.tex,v 1.6 2016/03/16 12:51:09 vollmer Exp $
%% -*- coding: utf-8 -*-
%%%%%%%%%%%%%%%%%%%%%%%%%%%%%%%%%%%%%%%%%%%%%%%%%%%%%%%%%%%%%%%%%%%%%%%%%%%%%%%

% In Bachelorarbeiten muss eine schriftliche Erklärung abgegeben werden.
% Hierin bestätigen die Studierenden, dass die Bachelorarbeit, etc.
% selbständig verfasst und sämtliche Quellen und Hilfsmittel angegeben sind. Diese Erklärung
% bildet das zweite Blatt der Arbeit. Der Text dieser Erklärung muss auf einer separaten Seite
% wie unten angegeben lauten.

\newpage
\thispagestyle{empty}
\begin{framed}
\begin{center}
\Large\bfseries Erklärung
\end{center}
\medskip
\noindent
Wir versicheren hiermit, dass unser Programmetwurf mit
dem Thema: \enquote{\Was} selbstständig verfasst wurde und keine anderen als die angegebenen Quellen und
Hilfsmittel benutzt wurden.

\vspace{3cm}
\noindent
\underline{\hspace{4cm}}\hfill\underline{\hspace{6cm}}\\
Ort~~~~~Datum\hfill Unterschrift\hspace{4cm}
\end{framed}

%%%%%%%%%%%%%%%%%%%%%%%%%%%%%%%%%%%%%%%%%%%%%%%%%%%%%%%%%%%%%%%%%%%%%%%%%%%%%%%
\endinput
%%%%%%%%%%%%%%%%%%%%%%%%%%%%%%%%%%%%%%%%%%%%%%%%%%%%%%%%%%%%%%%%%%%%%%%%%%%%%%%

% !TeX spellcheck = en_GB

\chapter*{Abstract}

Gleiche wie Zusammenfassung nur in Englisch


% !TeX spellcheck = de_DE

\chapter*{Zusammenfassung}


Warum/was?
Wie?
was kam raus?


\newpage
\tableofcontents           % Inhaltsverzeichnis hier ausgeben
\listoffigures             % Liste der Abbildungen
%\listoftables              % Liste der Tabellen
%\lstlistoflistings         % Liste der Listings
%\listofequations           % Liste der Formeln

% Jetzt kommt der "eigentliche" Text

%%%%%%%%%%%%%%%%%%%%%%%%%%%%%%%%%%%%%%%%%%%%%%%%%%%%%%%%%%%%%%%%%%%%%%%%%%%%%%%
%       Vorlage Abkürzungsverzeichnis
%%%%%%%%%%%%%%%%%%%%%%%%%%%%%%%%%%%%%%%%%%%%%%%%%%%%%%%%%%%%%%%%%%%%%%%%%%%%%%%

\chapter*{Abkürzungsverzeichnis}        
% chapter*{..} -->   keine Nummer, kein "Kapitel"
% Wird dadurch nicht ins Inhaltsverzeichnis eingetragen

% Damit das doch ins Inhaltsverzeichnis kommt
% \addcontentsline{toc}{chapter}{Akürzungsverzeichnis}   

%%%%%%%%%%%%%%%%%%%%%%%%%%%%%%%%%%%%%%%%%%%%%%%%%%%%%%%%%%%%%%%%%%%%%%%%%%%%%%%%
% Hier werden die Abkürzungen definiert
% \acro{Name}{Darstellung der Abkürzung}{Langform der Abkürzung}
 % Folgendes benutzen, wenn der Plural einer Abk. benöigt wird
 % \newacroplural{Name}{Darstellung der Abkürzung}{Langform der Abkürzung}
 % Wenn nicht benutzt, erscheint diese Abk. nicht in der Liste
%%%%%%%%%%%%%%%%%%%%%%%%%%%%%%%%%%%%%%%%%%%%%%%%%%%%%%%%%%%%%%%%%%%%%%%%%%%%%%%% 
\begin{acronym}[OPC-UA]

%.	
\acro{.NET}[.NET]{.NET von Mircorsoft}
%A
\acro{Add-On}[Add-On]{Plug-in, Software-Erweiterungen}
\acro{Agilent}[Agilent]{Agilent Technologies}
\acro{ARM}[ARM]{Advanced RISC Machine}
\acro{API}[API]{Application-Programming-Interfaces}
%B
\acro{bzw.}[bzw.]{beziehungsweise}
%C
\acro{CDS}[CDS]{Chromatograpy Data System}
\acro{CE}[CE]{Kapillarelektrophorese}
\acro{COM}[COM]{Component Object Model}
\acro{CRM}[CRM]{Customer Relationship Management}

\acro{CPU}[CPU]{Central Processing Unit}
%D
\acro{DSL}[DSL]{Domain Specific Language}
%E
\acro{ELK}[ELK]{Elasticsearch, Logstash und Kibana}
\acro{ETL}[ETL]{Extract, Transform and Load}
%F

%G

%H
\acro{HP}[HP]{Hewlett-Peckard}
\acro{HPLC}[HPLC]{Hochleistungsflüssigkeitschromatographen}
\acro{http}[http]{Hypertext Transfer Protocol}
%I
\acro{IoT}[IoT]{Internet of Things}
\acro{IP}[IP]{Internet Protocol}
%J
\acro{JSON}[JSON]{JavaScript Object Notation}
%K
%L
\acro{LC}[LC]{Flüssigkeitschromatographen}
\acro{LINQ}[LINQ]{Language Integrated Query}
%M
\acro{MIT}[MIT]{Massachusetts Institute of Technology}
%N
\acro{NoSQL}[NoSQL]{non-relational oder Not only SQL}
%O
%P
\acro{PHP}[PHP]{Hypertext Preprocessor}
\acro{Plug-in}[Plug-in]{Software-Erweiterungen}
\acro{PL29}[PL29]{Product Line 29}
%Q
%R
\acro{RESTful}[RESTful]{Representational State Transfer}
%S
\acro{SDK}[SDK]{Software Development Kit}
\acro{SID}[SID]{Software Informatik Division}
\acro{SQL}[SQL]{Structured Query Language/Strukturierte Abfrage Sprache}
%T
%U
\acro{UTF}[UTF]{Unicode Transformation Format}
%V
%W
%X
\acro{XML}[XML]{Extensible Markup Language}
%Y
\acro{YAML}[YAML]{YAML Ain't Markup Language}
%Z
\acro{z.B.}[z.B.]{zum Beispiel}
%123

%%%%%%%%%%%%%%%%%%%%%%%%%%%%%%%%%%%%%%%%%%%%%%%%%%%%%%%%%%%%%%%%%%%%%%%%%%%%%%
%       Wie im Text zitieren??
%%%%%%%%%%%%%%%%%%%%%%%%%%%%%%%%%%%%%%%%%%%%%%%%%%%%%%%%%%%%%%%%%%%%%%%%%%%%%%
% \ac{name}         Nur die Abkürzung wird referenziert
% \acs{name}        Langform mit Abkürzung in Klammer


\end{acronym}
              % Abkürzungsverzeichnis
\pagenumbering{arabic}

%%%%%%%%%%%%%%%%%%%%%%%%%%%%%%%%%%%%%%%%%%%%%%%%%%%
\chapter{Einleitung}
%%%%%%%%%%%%%%%%%%%%%%%%%%%%%%%%%%%%%%%%%%%%%%%%%%%
% Überschriften
%   \section{...}       Überschrift                 im Inhaltsverzeihcnis
%   \subsection{...}    Unterüberschrift            im Inhaltsverzeichnis
%   \subsubsection{...} Unterunterüberschrift       wird nicht mehr im Inhaltsverzeichnis angezeigt

%%%%%%%%%%%%%%%%%%%%%%%%%%%%%%%%%%%%%%%%%%%%%%%%%%%%%%%%%%%%%%%%%%%%%%%%%%%%%%%%%%%
\chapter{Clean Architecture}
Bei der Einrichtung des Projekts haben wir besonderen Wert auf die saubere Trennung der Schichten entsprechend der Clean Architecture gelegt.
Jede Schicht ist als eigenes Projekt im Solution-Explorer organisiert – beginnend bei 3\_Domain, über 2\_Application und 1\_Adapters bis hin zu Plugins.
Dadurch wird bereits auf Dateiebene erzwungen, dass die Abhängigkeiten nur „nach innen“ zeigen.

Die Domain-Schicht referenziert keine anderen Projekte. Die Application-Schicht referenziert ausschließlich 3\_Domain, und 1\_Adapters kennt
sowohl 2\_Application als auch 3\_Domain. Das Testprojekt und die Plugin-Schicht kennen alle anderen Projekte. Ein Rückgriff auf innere Schichten
von außen ist damit hoffentlich technisch ausgeschlossen. Damit soll garantiert werden, dass die Dependency Rule konsequent eingehalten wird.

Für die Verbindung der einzelnen Schichten verwenden wir Dependency Injection. Alle Use Cases und Services erhalten ihre Abhängigkeiten wie Repositories
über Interfaces, die in der Application-Schicht definiert sind. Die konkreten Implementierungen werden in der Plugin-Schicht (Program.cs)
registriert. So bleiben alle inneren Schichten unabhängig von konkreten Technologien und jederzeit testbar.
%%%%%%%%%%%%%%%%%%%%%%%%%%%%%%%%%%%%%%%%%%%%%%%%%%%%%%%%%%%%%%%%%%%%%%%%%%%%%%%%%%%
\section{Domain}
Die Domain-Schicht bildet das Herzstück der Anwendung und enthält ausschließlich fachliche Modelle und Logik. Hier werden zentrale Begriffe wie
Pflanze, Pflegeplan und Pflegeaufgabe als Entitäten oder Value Objects modelliert. Ziel dieser Schicht ist es, unabhängig von Technologien und
Implementierungsdetails zu bleiben.

In unserer Domain haben wir unter anderem folgende Konzepte definiert:
\begin{itemize}
    \item Plant und PlantType
    \item Careplan
    \item CareTasks in Form von RecurringCareTask, ScheduledCareTask mit entsprechenden CareTypes und CareTaskDetails wie WateringDetails, FertilizingDetails
    \item Location
\end{itemize}
Zusätzlich wurden in dieser Klasse weitere grundlegende Klassen wie Exeptions und eine CareTaskFactory erstellt. Die nach näherer Betrachtung
vermutlich in die Application-Schicht ausgelagert werden könnten.

Die grundlegendsten Funktionen und Logiken der hier definierten Klassen wird ebenfalls erstellt z. B. das Berechnen von IsDue() oder IsOverdue()
bei wiederkehrenden Aufgaben.

Die Domain kennt weder Datenbanken noch Schnittstellen – sie ist vollständig unabhängig und dadurch leicht testbar und wiederverwendbar.
%%%%%%%%%%%%%%%%%%%%%%%%%%%%%%%%%%%%%%%%%%%%%%%%%%%%%%%%%%%%%%%%%%%%%%%%%%%%%%%%%%%
\section{Application}
Die Application-Schicht stellt das Bindeglied zwischen der fachlichen Logik (Domain) und der Außenwelt dar. Unser Fokus lag darauf, hier klar
abgegrenzte Anwendungsfälle („Use Cases“) zu definieren, die konkrete Abläufe kapseln – wie etwa das Anlegen einer Pflanze oder das Durchführen
einer wiederkehrenden Pflegeaufgabe. Alle Abläufe, die ein Benutzer typischerweise in einer Anwendung auslöst, sollten hier als eigenständige
Klassen modelliert werden.

Wir haben uns bewusst gegen ein generisches Service-Konzept entschieden und stattdessen die Use-Case-Klassen als jeweils zuständig für genau
einen Anwendungsfall konzipiert. Das erhöht zwar die Anzahl der Klassen, sorgt aber für eine klare Trennung der Verantwortlichkeiten. Eine
Herausforderung bestand darin, die Zuständigkeiten sauber aufzuteilen. Wir haben komplexe Logik bewusst in die Adapter-Schicht verlagert und
sämtliche Ablaufentscheidungen aus den Use-Cases ferngehalten. So werden z. B. Benachrichtigungen bei überfälligen Aufgaben nicht automatisch
verschickt, sondern über den Use Case GetDueTasks lediglich zur Verfügung gestellt. Das macht die Anwendung flexibler – etwa für geplante
Erweiterungen wie eine mobile App oder eine Integration mit Sensoren.

Für die technische Entkopplung der Schicht setzen wir konsequent auf Interfaces für alle externen Abhängigkeiten. So verwendet z. B. CreatePlant
das IPlantRepository, ohne zu wissen, ob die Daten später aus einer SQLite-Datenbank, einem Webservice oder einer JSON-Datei stammen. Diese
Interfaces werden über Dependency Injection bereitgestellt, was die Testbarkeit erhöht und zukünftige Änderungen an der Persistenzschicht
erleichtert. Diese Interfaces werden in der Application-Schicht definiert.
%%%%%%%%%%%%%%%%%%%%%%%%%%%%%%%%%%%%%%%%%%%%%%%%%%%%%%%%%%%%%%%%%%%%%%%%%%%%%%%%%%%
\section{Adapter}
Die Adapter-Schicht dient in unserer Architektur als technische Brücke zwischen der Application-Schicht und der Außenwelt. Hier werden
beispielsweise Daten aus einer Datenbank geladen und in Form gebracht, sodass sie von der Application-Schicht genutzt
werden können. Hier würde typischerweise ebenfalls die bereitstellung von Endpunkten einer API stattfinden. Allerdings lag in unserem Projekt
der Fokus auf der Backend-seitigen Vorbereitung.

Die Implementierung von HTTP-Controllern – die in Clean Architecture ebenfalls in die Adapter-Schicht gehören – haben wir bewusst zurückgestellt. Zwar hätten wir mit .NET eine API-Anbindung relativ einfach realisieren können, jedoch hätten wir dafür zusätzliche Framework-Konventionen und externe Abhängigkeiten berücksichtigen müssen.

Stattdessen haben wir uns auf zwei Dinge konzentriert:
\begin{itemize}
    \item Ressourcen und Mapper: Wir haben erste Datenstrukturen entworfen, mit denen sich z. B. Pflegeaufgaben oder Pflanzen für eine spätere API-Ausgabe
strukturieren lassen. Dazu gehören einfache DTOs sowie Mapper, mit denen sich Domain-Modelle in transportfreundliche Objekte überführen lassen.
    \item Repository-Implementierungen: Ein zentraler Bestandteil dieser Schicht sind unsere konkreten Implementierungen der Repository-Interfaces. So
enthält die Adapter-Schicht beispielsweise die Klassen SqlPlantRepository, SqlCarePlanRepository und SqlLocationRepository, die den Datenzugriff
über Dapper und SQLite kapseln. Die Domain- und Application-Schicht bleiben dadurch vollständig unabhängig von der verwendeten
Datenbanktechnologie.
\end{itemize}

Durch diese Trennung bleibt unser System modular: Die Adapter-Schicht kann in einem nächsten Schritt um Controller oder ViewModel-Adapter
ergänzt werden, ohne dass Änderungen an der Fach- oder Anwendungslogik nötig sind.
%%%%%%%%%%%%%%%%%%%%%%%%%%%%%%%%%%%%%%%%%%%%%%%%%%%%%%%%%%%%%%%%%%%%%%%%%%%%%%%%%%%
\section{Plugin}
Die Plugin-Schicht enthält die konkreten externen Abhängigkeiten, die zum Starten und Ausführen der Anwendung nötig sind. In unserem Projekt
umfasst sie zwei zentrale Bestandteile:

\begin{itemize}
    \item Datenbankkonfiguration: Hier wird die SQLite-Verbindung eingerichtet. Die Adapter-Schicht nutzt Dapper, die Plugin-Schicht stellt
    dafür die konkrete Datenquelle bereit.
    \item Programmstart (Program.cs): In dieser Datei werden die Abhängigkeiten per Dependency Injection registriert und die Anwendung gestartet. Damit verbindet die Plugin-Schicht alle Komponenten zu einem lauffähigen System.
\end{itemize}

Wir haben darauf geachtet, alle externen Technologien klar auf diese Schicht zu beschränken. So bleibt der Rest der Anwendung unabhängig und leicht erweiterbar – z. B. für andere Datenbanken, Logging oder eine API-Anbindung.
% !TeX spellcheck = de_DE
\chapter{Umsetzung}
%%%%%%%%%%%%%%%%%%%%%%%%%%%%%%%%%%%%%%%%%%%%%%%%%%%%%%%
% Bilder einfügen und Referenzieren

%       \begin{figure}[]        Bild Umgebung 
%       \end{figure}
%       \ref{label}     wird im Text referenziert aber nur die nummer des Bildes
%                       will man "siehe Abbildung 1.2" dann "siehe Abbildung \ref{label}"
%       [p]             es gibt auch hier Parameter zur Ausgabe, mit p wird gezwungen spätestens auf der nächsten Seite das Bild/Tabelle auszugeben.

%       \includegraphics[]{}    eingindung des Bildes
%       [scale = "x"]     verändert die Größe des Bildes anhand des Faktors ( x = 0.5 halb so groß)
%       {img/Agilent_Long_Signature_RGB}    img Steht für den Ordner, das Bild wird ohne Datei Endung angegeben

%       \figcaption{\label{label} Bild Beschreibung}

\subsection{Wie referenziere ich ein Bild?}

Bla Bla Viel Text hier und da siehst du unser \ac{Agilent} Logo /ref{AgLog}.

\begin{figure}[h]
	\centering
	\includegraphics[scale=0.6]{img/dhbw-logo.png}
	\figcaption{\label{AgLog}Das ist unser Agilent Logo}
\end{figure}

% !TeX spellcheck = de_DE
\chapter{Ausblick}

Hier noch ein paar Zitate \cite{einstein} (normal), nur um zu testen ob das auch funktioniert \cite{dirac} \citet{NoSQL}\\

Und weil das so schön ist noch ein anderes \cite{knuthwebsite}.
Wie ihr seht verwende ich hier ein alphabetisch sortiertes Literaturverzeichnis
% !TeX spellcheck = de_DE
\chapter{Fazit}

%\include{content/kapitel6}

%%%%%%%%%%%%%%%%%%%%%%%%%%%%%%%%%%%%%%%%%%%%%%%%%%%%%%%%%%%%%%%%%%%%%%%%%%%%%%%%
% Ab hier beginnt der Anhang
%%%%%%%%%%%%%%%%%%%%%%%%%%%%%%%%%%%%%%%%%%%%%%%%%%%%%%%%%%%%%%%%%%%%%%%%%%%%%%%%
%\appendix
%\addcontentsline{toc}{chapter}{Anhang}
%\include{appendix}
%\addcontentsline{toc}{chapter}{Index}
%\printindex


%%%%%%%%%%%%%%%%%%%%%%%%%%%%%%%%%%%%%%%%%%%%%%%%%%%%%%%%%%%%%%%%%%%%%%%%%%%%%%%%
% Bibliothek wird hier eingebunden. 
\bibliography{bericht}
%%%%%%%%%%%%%%%%%%%%%%%%%%%%%%%%%%%%%%%%%%%%%%%%%%%%%%%%%%%%%%%%%%%%%%%%%%%%%%%%

\end{document}

%%%%%%%%%%%%%%%%%%%%%%%%%%%%%%%%%%%%%%%%%%%%%%%%%%%%%%%%%%%%%%%%%%%%%%%%%%%%%%%%
%%%%%%%%%%%%%%%%%%%%%%%%%%%%%%%%%%%%%%%%%%%%%%%%%%%%%%%%%%%%%%%%%%%%%%%%%%%%%%%%
% Nützliche Codes für den Content/Kapitel
%
%   wenn Text über den Rand geht kann man das mit sloppypar
%   \begin{sloppypar}
%       bla bla text...
%   \end{sloppypar}
%   
%   Kursiver Text:
%   \textit{...}
%
%
%
%
%