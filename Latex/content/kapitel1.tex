%%%%%%%%%%%%%%%%%%%%%%%%%%%%%%%%%%%%%%%%%%%%%%%%%%%
\chapter{Einleitung}
%%%%%%%%%%%%%%%%%%%%%%%%%%%%%%%%%%%%%%%%%%%%%%%%%%%
Die Idee zu diesem Projekt entstand aus einem ganz persönlichen Bedürfnis: Kira möchte langfristig ein System entwickeln, mit dem sie die
Pflanzen in ihrem hydroponischen Garten verwalten kann. Statt sich alleine durch ein großes Softwareprojekt zu kämpfen, haben wir die
Gelegenheit genutzt, diese Idee mit dem Programmentwurf zu verbinden – und damit die Grundlage für eine spätere, vollständige Anwendung
geschaffen.

Unser Ziel war es nicht, eine fertige App abzuliefern, sondern ein robustes, erweiterbares Backend zu entwickeln, das zentrale Funktionen wie
die Verwaltung von Pflanzen, Pflegeplänen und Pflegeaufgaben unterstützt. Dabei haben wir besonderen Wert auf eine saubere Strukturierung nach
den Prinzipien der Clean Architecture, testbare Komponenten und sinnvolle Domain-Modelle gelegt – alles mit dem Ziel, das System später
problemlos erweitern zu können.

Der Abschnitt zu Unit Tests soll doppelt gewertet werden.

Unsere Codebasis ist unter https://github.com/kira-isi/plant-care-backend öffentlich zugänglich.